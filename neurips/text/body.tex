\section{First section}
\label{body}

Organize the paper around a single core contribution with surgical precision. Do not add additional fluff.

Some general guidelines:

\textbf{Guideline \#1}: A clear new important technical contribution should have been articulated by the time the reader finishes page 3 (i.e., a quarter of the way through the paper).

\textbf{Guideline \#2}: Every section of the paper should tell a story. (Don't, however, fall into the common trap of telling the entire story of how you arrived at your results. Just tell the story of the results themselves.) The story should be linear, keeping the reader engaged at every step and looking forward to the next step. There should be no significant interruptions -- those can go in the Appendix; see below.

Aside from these guidelines, which apply to every paper, the structure of the body varies a lot depending on content. Important components are:

\begin{itemize}

    \item \textbf{Running Example}: When possible, use a running example throughout the paper. It can be introduced either as a subsection at the end of the Introduction, or its own Section 2 or 3 (depending on Related Work).
    \item \textbf{Preliminaries}: This section, which follows the Introduction and possibly Related Work and/or Running Example, sets up notation and terminology that is not part of the technical contribution. One important function of this section is to delineate material that's not original but is needed for the paper. Be concise -- remember Guideline \#1.
    \item \textbf{Content}: The meat of the paper includes algorithms, system descriptions, new language constructs, analyses, etc. Whenever possible use a "top-down" description: readers should be able to see where the material is going, and they should be able to skip ahead and still get the idea.
\end{itemize}

\subsection{method explanation}
Method/model description with technical section somewhere with math details.

\subsection{more sections of the paper}
Explain what is being done in the section. Explain what the core challenges are. Explain what a baseline approach is or what others have done before. Motivate and explain what I'm doing

\subsection{more sections of the paper}
Explain what is being done in the section. Explain what the core challenges are. Explain what a baseline approach is or what others have done before. Motivate and explain what I'm doing